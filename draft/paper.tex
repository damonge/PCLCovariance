\documentclass[useAMS,usenatbib]{mn2e}
\bibliographystyle{mn2e}
\pdfoutput=1
\pdfminorversion=5

%\usepackage{widetext}
\usepackage{graphicx}
\usepackage{dcolumn}
\usepackage{bm}
\usepackage{amssymb,amsmath,bm}
\usepackage{color}
\usepackage[dvipsnames]{xcolor}
\usepackage[colorlinks,linkcolor=red,citecolor=blue,urlcolor=blue ]{hyperref}
\usepackage{multirow}
\usepackage[utf8]{inputenc}
\usepackage{balance}
\usepackage{enumitem}
\usepackage{lipsum}
\newcommand{\nv}{\hat{\bf n}}
\newcommand{\kalo}{Karhunen-Lo\`{e}ve~}
\newcommand{\cfh}{CFHTLenS~}
\newcommand{\jcap}{JCAP}
\newcommand{\mnras}{MNRAS}
\newcommand{\aap}{A\&A}
\newcommand{\aaps}{A\&AS}
\newcommand{\apjs}{ApJS}
\newcommand{\apj}{ApJ}
\newcommand{\apjl}{ApJL}
\newcommand{\prd}{Phys.~Rev.~D}
\newcommand{\prl}{Phys.~Rev.~Lett.}
\newcommand{\aj}{Astron. Journal}
\newcommand{\pasp}{Publications of the ASP}
\newcommand{\nar}{New Astronomy Review}
\newcommand{\procspie}{Proceedings of the SPIE}
\newcommand{\physrep}{Physics Reports}

\newcommand{\todo}[1]{{\bf TODO: #1}}
%\newcommand{\todo}[1]{$\,$}

\newcommand{\pcl}[3]{\hat C_{#1}^{#2 #3}}
\newcommand{\avg}[1]{\langle #1 \rangle}
\newcommand{\fsky}{f_{\mbox{sky}}}
\newcommand{\clth}{C_l^{th}}
\newcommand{\clf}{C_l^{fore}}


\title[Covariance]{Covariance}
\author[Carlos Garc\`{i}a Garc\`{i}a]{Carlos Garc\`{i}a Garc\`{i}a$^1$\thanks{carlosgarcia@iff.csic.es}\\
$^{1}$Oxford Astrophysics, Department of Physics, Keble Road, Oxford, OX1 3RH, UK
}

\begin{document}
  \date{\today}
  \pagerange{1--18} \pubyear{2019}
  \maketitle

\begin{abstract}
\end{abstract}

\begin{keywords}
  cosmology: large-scale structure of the Universe -- methods: data analysis
\end{keywords}

\section{Introduction}
    
\section{Analytical Covariance}
\begin{itemize}
\item Naive Covariance
  \begin{equation}
    C^{abcd} = \frac{\pcl{l}ad \pcl{l}bc + \pcl{l}ac \pcl{l}bd}{f_{sky} (2l +
      1) \Delta l} \delta_{ll'}
  \end{equation}
\item Approximation 1
\item Approximation 2
\end{itemize}

\section{Results}
For both flat and curved sky we should discuss:
\begin{itemize}
\item Simulations (sims, methods)
\item Approximation 1 - spin 0
\item Approximation 1 - spin 2
\item Approximation 2
\item Naive
\end{itemize}

For both flat and curved plots:
\begin{itemize}
\item Cov Naive
\item Cov approx 1
\item Cov approx 2
\item Cov approx 2 + deprojection
\end{itemize}

Plots:
\begin{itemize}
\item Plot $\chi^2$
\item KS($\chi^2$)
\item Eigenvalues
\end{itemize}

In this section we will show the performance of each approximation and up to
which point they hold. We will compare the analytical covariance matrices with
those from simulated power spectra. 

The simulations were generated as follow. First, we starts from some realistic
large scale structure power spectra ($\clth$) and mask, which determines the
number of modes. To this theoretical power spectra, in some cases, we
added some foregrounds to study their impact on the approximated covariances.
The mock power spectra was computed as follows:
\begin{eqnarray}
  \clf = A (l + 1)^\beta\,,
\end{eqnarray}
where $A$ was fixed so that $\clf/\clth = 0.1$ at $l=400$ and $\beta$ was
randomly chosen from $U[-3, -1)$ for large scale effects and fixed to $\beta =
0$ for small scale ones.

The resulting power spectra was then used to generate a map of the sky and
two random fields (one of spin-0 and other of spin-2), in turn. Their
correlations give  are used to obtain the simulated power spectra. In order to
find the real power spectra, it is necessary to subtract the contribution of
the mask, which mixes the closest modes:
\begin{equation}
  C^{obs}_l = \sum_{l'} M_{ll'} C_{l'}\,
\end{equation}
where $M$ depends on the mask. In general, $M$ is not invertible and, in order
to be able to recover the true power spectra, $C_l$, one needs to bin the
$l$-space. The bin width is given by the characteristics of the window
function and must be, approximately, of the size of the range of the mixed
modes. That way, each $\tilde C_l$ is almost independent of the others
$\tilde C_{l'}$, and $M$ is invertible.

The code to generate simulated power spectra and analyze the posterior
results is publicly available in
\url{https://github.com/damonge/PCLCovariance} and uses
NaMaster~\cite{2018arXiv180909603A}, in order to compute the necessary fields
and their correlations. 


\section{Discussion}\label{sec:discussion}


\section*{Acknowledgements}
We would like to thank Eva-Maria Mueller for useful discussion. CGG is
supported by AYA2015-67854-P from the Ministry of Industry, Science and
Innovation of Spain and the FEDER funds and by the Spanish grant, partially
funded by the ESF, BES-2016-077038.

\setlength{\bibhang}{2.0em}
\setlength\labelwidth{0.0em}
\bibliography{paper.bib}

\end{document}
