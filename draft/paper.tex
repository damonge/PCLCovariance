\documentclass[useAMS,usenatbib]{mn2e}
\bibliographystyle{mn2e}
\pdfoutput=1
\pdfminorversion=5

%\usepackage{widetext}
\usepackage{graphicx}
\usepackage{dcolumn}
\usepackage{bm}
\usepackage{amssymb,amsmath,bm}
\usepackage{color}
\usepackage[dvipsnames]{xcolor}
\usepackage[colorlinks,linkcolor=red,citecolor=blue,urlcolor=blue ]{hyperref}
\usepackage{multirow}
\usepackage[utf8]{inputenc}
\usepackage{balance}
\usepackage{enumitem}
\usepackage{lipsum}
\newcommand{\nv}{\hat{\bf n}}
\newcommand{\kalo}{Karhunen-Lo\`{e}ve~}
\newcommand{\cfh}{CFHTLenS~}
\newcommand{\jcap}{JCAP}
\newcommand{\mnras}{MNRAS}
\newcommand{\aap}{A\&A}
\newcommand{\aaps}{A\&AS}
\newcommand{\apjs}{ApJS}
\newcommand{\apj}{ApJ}
\newcommand{\apjl}{ApJL}
\newcommand{\prd}{Phys.~Rev.~D}
\newcommand{\prl}{Phys.~Rev.~Lett.}
\newcommand{\aj}{Astron. Journal}
\newcommand{\pasp}{Publications of the ASP}
\newcommand{\nar}{New Astronomy Review}
\newcommand{\procspie}{Proceedings of the SPIE}
\newcommand{\physrep}{Physics Reports}

\newcommand{\todo}[1]{{\bf TODO: #1}}
%\newcommand{\todo}[1]{$\,$}


\title[Covariance]{Covariance}
\author[C. Garc\'{i}a Garc\'{i}a et al.]{Carlos Garc\'{i}a Garc\'{i}a$^1$\thanks{carlosgarcia@iff.csic.es}, David Alonso$^2$\thanks{david.alonso@physics.ox.ac.uk}, Emilio Bellini$^2$\thanks{emilio.bellini@physics.ox.ac.uk}\\
$^{1}$IFF\\
$^{2}$Oxford Astrophysics, Department of Physics, Keble Road, Oxford, OX1 3RH, UK
}

\begin{document}
  \date{\today}
  \pagerange{1--18} \pubyear{2019}
  \maketitle

\begin{abstract}
\end{abstract}

\begin{keywords}
  cosmology: large-scale structure of the Universe -- methods: data analysis
\end{keywords}

\section{Introduction}
    
\section{Analytical Covariance}
This section presents the analytical calculation of the pseudo-$C_\ell$ covariance. The treatment and approximations used are similar to those presented in \todo{cite} for spin-0 quantities, in \todo{cite} for particular spin-0 and spin-2 correlations and in \todo{cite} for the 3D density field. We reproduce the formalism here in order to provide a complete set of formulas applicable to any cross-correlation in both curved and flat skies.



\begin{itemize}
\item Naive Covariance
\item Approximation 1
\item Approximation 2
\end{itemize}

\section{Results}
For both flat and curved sky we should discuss:
\begin{itemize}
\item Simulations (sims, methods)
\item Approximation 1 - spin 0
\item Approximation 1 - spin 2
\item Approximation 2
\item Naive
\end{itemize}

For both flat and curved plots:
\begin{itemize}
\item Cov Naive
\item Cov approx 1
\item Cov approx 2
\item Cov approx 2 + deprojection
\end{itemize}

Plots:
\begin{itemize}
\item Plot $\chi^2$
\item KS($\chi^2$)
\item Eigenvalues
\end{itemize}

\section{Discussion}\label{sec:discussion}


\section*{Acknowledgements}
  
\setlength{\bibhang}{2.0em}
\setlength\labelwidth{0.0em}
\bibliography{paper}


\end{document}
